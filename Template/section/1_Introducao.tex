\section{Introdução}

Introdução...

\begin{itemize}
    \item Exemplo do uso de Figura~\ref{fig:name}.
    
    \begin{figure}[!htb]
    \centering
    \includegraphics[scale=.1]{_fig/anoti.png}
    \caption[Nome na tabela de figuras.]{Nome no caption da figura.}
    \label{fig:name}
    \end{figure}

    \item Exemplo do uso de citação:
    \begin{itemize}
        \item Direta: \citeonline{Silva2019}
        \item Indireta: \cite{Silva2019}
    \end{itemize}
    
    \item Exemplo de Tabela~\ref{tab:classifiersCKC}. Criado com o auxilio do \textit{Tables Generator}\footnote{\url{tablesgenerator.com/latex_tables}}.
    \begin{table}[!htp]
\centering
\caption[Nome na lista]{Nome no caption}
\label{tab:classifiersCKC}
\begin{tabular}{l|c|c|c|c|c||c|c} \hline \hline
 & 
\multicolumn{7}{c}{ {\cellcolor[HTML]{68CBD0}\textbf{Tamanho da Amostra}} }    \\ \hline
\cellcolor[HTML]{68CBD0} \textbf{Classificador} & 
\cellcolor[HTML]{68CBD0} \textbf{$1 \cdot 10^1$}  & 
\cellcolor[HTML]{68CBD0} \textbf{$1 \cdot 10^2$}  & 
\cellcolor[HTML]{68CBD0} \textbf{$1 \cdot 10^3$}  & 
\cellcolor[HTML]{68CBD0} \textbf{$1 \cdot 10^4$}  & 
\cellcolor[HTML]{68CBD0} \textbf{$1 \cdot 10^5$}  &
\cellcolor[HTML]{68CBD0} \textbf{$\bar{\mu}$} &
\cellcolor[HTML]{68CBD0} \textbf{$\mu$}\\ \hline \hline
Na\"ive-Bayes         & 0,156 & 0,710 & 0,793 & 0,797 & 0,805  & 0,804 & 0,652 \\ \hline
Bayes-Net         & 0,038 & 0,423 & 0,639 & 0,722 & 0,743  & 0,739 & 0,513 \\ \hline
kNN        & 0,675 & 0,874 & \textbf{0,967} & \textbf{0,967} & 0,971 & 0,970 & 0,890\\ \hline
\textbf{MLP-LM} & \textbf{0,697} & \textbf{0,895} & 0,953 & \textbf{0,967} & \textbf{0,972} & \textbf{0,971} & \textbf{0,896} \\ \hline
MLP-RPROP     & 0,423 & 0,829 & 0,929 & 0,958 & 0,968  & 0,967 & 0,821 \\ \hline
DecisionTree         & 0,493 & 0,790 & 0,922 & \textbf{0,967} & 0,970  & 0,969 & 0,828 \\ \hline
RandomForest         & 0,350 & 0,854 & 0,947 & \textbf{0,967} & \textbf{0,972} & \textbf{0,971} & 0,818\\ \hline 
RandomForest         & 0,350 & 0,854 & 0,947 & \textbf{0,967} & \textbf{0,972} & \textbf{0,971} & 0,818\\ \hline \hline

\end{tabular}
\end{table}
    
    \item Exemplo de aplicação de Definição~\ref{def:pixel}
    
    \begin{definition}\label{def:pixel}[Pixel] 
    Um pixel $P_i=\{r_i,g_i,b_i\} $ é constituído por uma tupla no espaço de cores RGB, onde $r_i$ é a componente do eixo vermelho, $g_i$ é a componente do eixo verde e $b_i$ é a componente do eixo azul.
    \end{definition}
    
    \item Exemplo de Equação~\eqref{eq:pesos}
    
    \begin{equation}\label{eq:pesos}
    P_i' = (r_i', g_i', b_i');~r_i' = g_i' = b_i' = r_i - \frac{g_i + b_i}{2}
    \end{equation}
    
    \item Exemplo de Algoritmo~\ref{alg:2pla}
    
    \begin{algorithm}[!htb]
    \nonl \rule[0.5ex]{410px}{0.5pt}\\
    \SetKwFunction{media}{Média}
    \SetKwInOut{Input}{Entrada}
    \SetKwInOut{Output}{Saída}
    \Input{Prova1 $p1$, Prova2 $p2$, Resultado $r$.}
    \Output{Valor fatorial.}
    \nonl \rule[0.5ex]{410px}{0.5pt}\\
    \tcc{Somatório}
    \media $\leftarrow \langle p1,p2 \rangle$\;
     \If{$\media > 5,9$ }{
        $r \leftarrow \langle$ ``Aprovado''$\rangle$\  \;
     }\Else{
        $r \leftarrow \langle$ ``Reprovado''$\rangle$\  \;
     }
    \Return $r$\;
    \nonl \rule[0.5ex]{410px}{0.5pt}\\
    \caption{Disciplina.}
    \label{alg:2pla}
    \end{algorithm}

\end{itemize}






\subsection{Motivação e contexto}

Motivação e Contexto...

\subsection{Objetivo}

Objetivos ...

\subsection{Organização da monografia}
Esse documento está organizado em XX capítulos.
Além desse, os demais capítulos incluem o seguinte conteúdo:
\begin{itemize}
    \item O Capítulo~\ref{sec:background} 
    \item O Capítulo~\ref{sec:methods} 
    \item O Capítulo~\ref{sec:exp_result} 
    \item O Capítulo~\ref{sec:conclusion} 
\end{itemize}

