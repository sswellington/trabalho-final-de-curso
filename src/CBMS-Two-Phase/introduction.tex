\section{Introduction}

Photographs of ulcers in lower limbs have the potential for supporting dermatologists analyses, as they include valuable information regarding the wounded area~\cite{Bates2016} and tissue characterization~\cite{Blanco2016,Todd2018}.
Image obtaining is a non-intrusive process, \textit{e.g.}, smartphone-based image recording~\cite{Deserno2018,Goyal2018}, while digital photo management can be carried out by non-complex engines in comparison to radiological systems~\cite{Gillies2015}.
Such advantages favor the exponential increase in the number of annotated photos digitally stored, which gives room for the design of new Computer-Aided Diagnosis (CAD) tools~\cite{Chino2018,Deserno2018}.
In contrast to pattern recognition tasks that aims at classifying and reporting skin abnormalities~\cite{Zahia2018}, the analysis of chronic ulcers are rather related to the \textit{segmentation} of wounded tissues so that they can be further integrated with patient data, previous diagnoses, or even radiomics systems~\cite{Gillies2015}.

Challenges regarding the design of segmentation methods for chronic ulcers in lower limbs consist of \textit{(i)}~characterizing particular types of tissues and 
\textit{(ii)}~quantifying tissue areas within and around wounded regions~\cite{Pereyra2014,Todd2018}.
Typically, CAD methods are based on either \textit{supervised} or \textit{unsupervised} learning paradigms.
Supervised approaches heavily rely on classification algorithms for the separation of  healthy skin pixels from those \textit{within} the ulcered region.
Godeiro et al.~\cite{Godeiro2018} use a color space reduction for the extraction of wounded regions according to their most dominant tissues at $94\%$ Dice Coefficient ratio.
Mukherjee et al.~\cite{Mukherjee2014} apply low-level color extractors and Support-Vector Machines for the labeling of chronic wounds into five classes at $87.61\%$ hit ratio.
Blanco et al.~\cite{Blanco2016} and Chino et al.~\cite{Chino2018} use the same set of five labels, but apply a divide-and-conquer strategy in which wounds are segmented with the support of superpixel construction methods~\cite{Achanta2012,Barbieri2015}.
Such methods divide an image into regions of interest with well-defined borders that are further extracted by feature engineering methods, as MPEG-7 descriptors~\cite{Blanco2016} or Bag-of-Signatures~\cite{Chino2018}. 
Extracted features are labeled by RandomForest in up to $89.87\%$ accuracy.

Deep-learning models combine both image extraction and data classification into a single package, which removes the need for feature engineering methods.
Goyal et al.~\cite{Goyal2018} use a deep network for the segmentation of diabetic ulcers into three groups at $92.5\%$ accuracy ratio, while Zahia et al.~\cite{Zahia2018} define a convolutional network for segmenting pressure ulcers at $91.3\%$ Dice Coefficient ratio.
The main drawback with \textit{supervised-only} approaches is they do not differ all tissue combinations within and around the wound by assuming a fixed and discrete set of labels (\textit{e.g.}, ``granulation'' or ``slough'') for splitting the image into ulcer and skin.

On the other hand, unsupervised methods describe chronic wounds by clustering similar regions into single tissue pieces, whereas ulcer borders are kept \textit{around} the wounded tissues~\cite{Dhane2017,Bates2016}.
Maity et al.~\cite{Maity2018} surveyed Region Growing, $k$-Means, Fuzzy $C$-means, and Gaussian Mixture Model algorithms for the analysis of venous ulcers, and results indicated Gaussian Mixture Model segmented chronic wounds at $95.91\%$ Sensitivity ratio.
Likewise, Dhane et al~\cite{Dhane2017} described a new fuzzy method for ulcer segmentation with up to $91.5\%$ accuracy ratio. 
Clustering approaches provide the basis for measuring the size, volume, and stage of the wounds, which enhances final diagnoses~\cite{Bates2016,Todd2018}.
However, as computational complexity of \textit{unsupervised-only} methods is gradually affected by the number of regions of interest, clustering routines can become time and power consuming, and, consequently, unsuitable for personal computers and battery-bounded devices~\cite{Deserno2018}. 

Aiming at surpassing those drawbacks, we propose a new approach that combines the best of supervised and unsupervised paradigms.
Our method, named \system (\textit{\textbf{T}wo-\textbf{P}hase \textbf{L}earning \textbf{A}pproach}), is a two-phase approach that removes points of no-interest (\textit{e.g.}, background area) for reducing the number of candidate regions to be clustered into different tissue pieces.
Given a wound photo recorded by a fixed protocol, \system first phase performs a pixelwise classification of potential points of interest, while pre-processing filters smooth lower limb borders.
\system second phase processes the cleaned data by using a divide-and-conquer strategy. 
It builds upon SLIC~\cite{Achanta2012} superpixel construction algorithm for dividing the lower limb into regions of interest with well-defined borders, and clusters the superpixels by taking advantage of the similarity-based DBSCAN algorithm~\cite{Kriegel2017}.

Since both \system phases are parametric, we discuss the tuning of pixel classification and superpixel clustering algorithms by using a real and annotated set of dermatological wounds.
Empirical evaluations on representative samples up to $100{,}000$ points showed a compact Multi-Layer Perceptron with Levenberg--Marquardt training method outperformed other classifiers as \system first phase.
We also evaluated DBSCAN performance with regards to five distance functions ($L_1$, $L_2$, $L_\infty$, Canberra, and BrayCurtis) and results indicated $L_1$ function generates fewer groups in comparison to the competitors, and the number of clusters is an exponential decay to the similarity ratio.
Accordingly, we used the elbow criterion~\cite{Jolliffe2011} as the $L_1$-based DBSCAN threshold for tuning \system second phase.
Finally, we performed an experiment with fine-tuned \system on real images and results showed tissues were segmented at a low error ratio.
In summary, the main contributions of this study are as follows:

\begin{itemize}
\item \underline{Wound segmentation}: We designed a two-phase approach (\system) that combines supervised and unsupervised learning for the segmentation of chronic wounds.

\item \underline{\system tuning}: We tuned our approach through empirical evaluations on a real set of ulcers.
Results indicated MLP, $L_1$ and the elbow criterion as suitable \system parameters.
\end{itemize}

The remainder of this paper is structured as follows. 
Section~2 introduces the basic concepts of the paper and the main related work. 
The proposed method is described in Section~3. 
Section~4 presents the experiments and discusses the results. 
Finally, Section~5 presents the conclusions and future work.

